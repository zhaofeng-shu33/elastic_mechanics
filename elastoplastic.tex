\documentclass{article}
\usepackage{amsmath}
\usepackage{bm}
\usepackage{color}
\usepackage{graphicx}
\usepackage{epstopdf}
\usepackage{amsfonts}
\usepackage{bm}
\begin{document}
We give the constitutive model for non-linear material behaviour in small deformation with von-Mises criterion yield function.
The flow potential $F$ is given by:
\[
F=f-k
\]
where
\[
f=\frac{1}{2}s_{ij}s_{ij}
\]
in tensor nototation, and 
\[
k=\frac{1}{3}\sigma_{y}^2
\]
The consistency condition (in plastic deformation period) requires that:
$F=0$.
Therefore the total derivative $dF=0$
and we can get (in tensor notation)
\begin{equation}\label{eq:Consistency}
\frac{\partial{f}}{\partial \sigma_{ij}}d\sigma_{ij}-\frac{2}{3}\sigma_{y}\frac{d\sigma_y}{d\bar{\epsilon}_p}d\bar{\epsilon}_p=0
\end{equation}

where $\bar{\epsilon}_p$ is the acumulated plastic strain.
By flow rule we can get 
\[
d \epsilon^p=d\lambda \frac{\partial f}{\partial{\bm{\sigma}}}
\]
we also have \[
d \bar{\epsilon}_p=\sqrt{\frac{2}{3}}||d\epsilon^p||_f
\]
where $||\cdot||_f$ denotes the Frobinious norm of a rank 2 tensor.
By the expresion of $f$,
we have
\[
\frac{\partial{f}}{\partial \sigma_{ij}}=s_{ij}
\]
therefore,
\[
d \bar{\epsilon}_p=\sqrt{\frac{2}{3}}d\lambda||\bm{s}||_f
\]
Since $f-k=0$ we have $||\bm{s}||_f=\sqrt{\frac{2}{3}}\sigma_y$
In conclusion we have:
\begin{equation}
d \bar{\epsilon}_p=\frac{2}{3}\sigma_yd\lambda
\end{equation}
Define the hardening modulus: $E_p=\frac{d\sigma_y}{d\bar{\epsilon}_p}$
From (\ref{eq:Consistency}) we get:
\begin{equation}\label{eq:xxx}
s_{ij}d\sigma_{ij}-\frac{4}{9}\sigma_y^2E_pd\lambda=0
\end{equation}
In small deformation, the strain increment can be decomposed additively as: 
\[
d\epsilon_{ij}=d \epsilon_{ij}^e+d \epsilon_{ij}^p
\]
By generalized Hook's Law:
\begin{align*}\notag
d\sigma_{ij}=&D^e_{ijkl}(d\epsilon_{kl}-d\epsilon_{kl}^p)\\
=&D^e_{ijkl}(d\epsilon_{kl}-s_{kl}d\lambda )
\end{align*}
Combining this equation with (\ref{eq:xxx}), we can solve out:
\[
d\lambda=\frac{s_{ij}D_{ijkl}^e d \epsilon_{kl}}{s_{ij}s_{kl}D^e_{ijkl}+\frac{4}{9}\sigma_y^2 E_p}
\]
We replace $d\lambda$ in $d\sigma_{ij}=D^e_{ijkl}(d\epsilon_{kl}-s_{kl}d\lambda )$ with the above expression and get the non-linear relation between stress and strain in elastoplastic deformation:
\[
d\sigma_{ij}=D^{ep}_{ijkl}d \epsilon_{kl}
\]
where $D^{ep}_{ijkl}=D^e_{ijkl}-D^p_{ijkl}$. $\bm{D}^{ep}$ is called  elastoplastic tangent modulus and 
\[
D^p_{ijkl}=\frac{s_{mn}D^e_{ijmn}s_{qr}D^e_{qrkl}}{s_{mn}s_{qr}D^e_{qrmn}+\frac{4}{9}\sigma_y^2 E_p}
\]
Since (in 3D)
\[
D^e_{qrmn}=\lambda \delta_{qr}\delta_{mn}+G(\delta_{qm}\delta_{rn}+\delta_{qn}\delta_{rm})
\]
The denominator can be simplified to (use $\frac{1}{2}s_{ij}s_{ij}=\frac{\sigma_y^2}{3}$):
\begin{align*}\notag
s_{mn}s_{qr}D^e_{qrmn}+\frac{4}{9}\sigma_y^2 E_p=&\lambda(s_{mm}s_{qq})+2G s_{mn}s_{mn}+\frac{4}{9}\sigma_y^2 E_p\\
=&\frac{4\sigma_y^2}{9}(3G+E_p)
\end{align*}
Similarly, the numerator can be simplified to
\[
s_{mn}D^e_{ijmn}s_{qr}D^e_{qrkl}=4G^2s_{ij}s_{kl}
\]
As a result, we get:
\begin{equation}
D^p_{ijkl}=\frac{9G^2 s_{ij}s_{kl}}{\sigma_y^2(3G+E_p)}
\end{equation}
\end{document}
