\documentclass{article}
\usepackage{amsmath}
\usepackage{bm}
\usepackage{color}
\usepackage{graphicx}
\usepackage{epstopdf}
\usepackage{amsfonts}
\usepackage{bm}
\begin{document}
\section{Virtual Displacement Principle}
From Equilibrium Equation (tensor notation):
\[
\sigma_{ij,j}+f_i=0 \text{ in } V
\]
stress BC:
\[
\sigma_{ij}n_j=T_i \text{ in } S_{\sigma}
\]
Multiplying the above two equations with virtual displacement
$\delta{u}_i$ and integral in the corresponding region, we get
\begin{equation}
\int_V \delta{u_i}(\sigma_{ij,j}+f_i)dV-\int_{S_{\sigma}} \delta{u_i}(\sigma_{ij}n_j-T_i)dS=0
\end{equation}
We integrate the following expression by part($\delta u_i:=(\delta u)_i$):
\begin{align*}
\int_V \delta{u_i}\sigma_{ij,j}dV=& \int_V (\delta u_i \sigma_{ij} )_j dV-\int_V \frac{1}{2}(\delta u_{i,j}+\delta u_{j,i})\sigma_{ij} dV\\
=&\int_{S_{\sigma}} \delta u_i \sigma_{ij} n_j dS-\int_V \delta \epsilon_{ij} \sigma_{ij} dV
\end{align*}
It follows:
\begin{equation}
\int_V \delta{\epsilon_{ij}}\sigma_{ij}dV=\int_V \delta u_i f_i dV+\int_{S_{\sigma}} \delta{u_i}T_idS
\end{equation}
For linear elasticity:
\begin{equation}\label{eq:LE}
\int_V \delta{\epsilon_{ij}}D_{ijkl}^e \epsilon_{kl}dV=\int_V \delta u_i f_i dV+\int_{S_{\sigma}} \delta{u_i}T_idS
\end{equation}
$V=\cup V_i$ where $V_i \cap V_j =\emptyset$ for $i\neq j$.
Each $V_i$ is a cell which contains K nodes (vertex). The degree of freedom used on each cell equals the dimension of the problem times the number of nodes per cell.
Consider a particular cell $V^*$, on $V^*$ we have:
\[
u_i=U_{ti}\phi_{ti}
\]
Take $\delta u_i=\delta_{ii^*}\phi_{ri^*}$($r,i^*$ is prescribed, corresponding to $\delta$), we have:
\begin{align*}
\delta{\epsilon_{ij}}=&\frac{1}{2}(\delta u_{i,j}+\delta u_{j,i})\\
=&\frac{1}{2}(\delta_{ii^*}\phi_{ri^*,j}+\delta_{ji^*}\phi_{ri^*,i})
\end{align*}
and
\begin{align*}
{\epsilon_{kl}}=&\frac{1}{2}(\delta u_{k,l}+\delta u_{l,k})\\
=&\frac{1}{2}(U_{tk}\phi_{tk,l}+U_{tl}\phi_{tl,k})
\end{align*}
For isotropic material $D_{ijkl}^e=D_{ijlk}^e$,then
\begin{align*}
D_{ijkl}^e\epsilon_{kl}=&\frac{1}{2}(D_{ijkl}^eU_{tk}\phi_{tk,l}+D_{ijlk}^eU_{tk}\phi_{tk,l})\\
=&D_{ijkl}^eU_{tk}\phi_{tk,l}
\end{align*}
The above relation also holds for $D^{ep}_{ijkl}=D^e_{ijkl}-D^p_{ijkl}$
For given $r,i^*,t,k$, we have:
\begin{align*}
\delta \epsilon_{ij}D_{ijkl}^e\epsilon_{kl}=&\delta_{ii^*}\phi_{ri^*,j}D_{ijkl}^e \phi_{tk,l} U_{tk}\\
=&\left(\delta_{ii^*}\phi_{ri^*,j}(\lambda\delta_{ij}\phi_{tk,k}+G(\delta_{ik}\phi_{tk,j}+\delta_{jk}\phi_{tk,i}))\right)U_{tk}\\
=&(\lambda \phi_{ri^*,i^*}\phi_{tk,k}+G \phi_{ri^*,k}\phi_{tk,i^*}+G\delta_{i^*k}\phi_{ri^*,j}\phi_{tk,j})U_{tk}
\end{align*}
Notice the last term sums over $j$ when $i^*=k$.
For $D^{ep}_{ijkl}$, we revise the above by $\delta \epsilon_{ij} D^p_{ijkl} \epsilon_{kl}$ when the yielding condition is reached:
\begin{equation}
\delta \epsilon_{ij} D^p_{ijkl} \epsilon_{kl}=\phi_{ri^*,j}D^p_{i^*jkl}\phi_{tk,l} U_{tk}
\end{equation}
The summation is over $ j$ and $l$.
From (\ref{eq:LE}), we get for plastic case that:
\begin{equation}
\int_V \delta{\epsilon_{ij}}(D_{ijkl}^{ep} \Delta \epsilon_{kl}+\sigma^n_{ij})dV=\int_V \delta u_i f_i dV+\int_{S_{\sigma}} \delta{u_i}T_idS
\end{equation}
Then we get that $\sigma^n_{ij}$ contributes to the RHS:
\begin{equation}
-\int_V \delta{\epsilon_{ij}}\sigma^n_{ij}=-\int_V  \phi_{ri^*,j}\sigma_{i^*j}^n dV
\end{equation}
For axisymmetric problem, the displacement component vanishes along the $\hat{\theta}$, therefore the problem reduces to 2D.
From (\ref{eq:LE}) we change the integration to cylinderal representation and with $u_{\theta}\equiv 0$ in mind and ignoring $\bm{f}$, we get
\begin{equation}
\int_{\Sigma}\delta{\epsilon}_{ij}\sigma_{ij}rdrdz=\int_{L} \delta u_r T_r+\delta u_z T_z dz
\end{equation}
where $\Sigma$ is a section of revolution plane, and $L$ is $\partial\Sigma$ minus the revolution axis.To solve the equivalent 2D problem, we need mesh $\Sigma$ and apply Dirichlet BC on the revolution axis for $u_r$.
For the LHS, we can further simply it using matrix form:
\begin{align}
\delta{\epsilon}_{ij}\sigma_{ij}=&(\delta \bm{\epsilon})^T D \bm{\epsilon}\\
=&(\delta \bm{u})^T (\bm{B}\bm{D}\bm{B})\bm{u}
\end{align}
where $u=(u_r,u_z)^T$ and 
\[
B= \left(\begin{matrix}
\frac{\partial}{\partial r} & 0\\
0 & \frac{\partial}{\partial r}\\
\frac{1}{r} & 0\\
\frac{\partial}{\partial z}&\frac{\partial}{\partial r}\\
\end{matrix}\right),D=\left(\begin{matrix}
\lambda+2G & \lambda&\lambda&0\\
\lambda & \lambda+2G&\lambda&0\\
\lambda & \lambda&\lambda+2G&0\\
0&0&0&G\\
\end{matrix}\right)
\]
In real calculation, there will be a term $\frac{1}{r}$ for integration, and the mesh grid can not be too refined near the revolution axis.
\end{document}
