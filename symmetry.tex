\documentclass{article}
\usepackage{amsmath}
\usepackage{bm}
\usepackage{color}
\usepackage{graphicx}
\usepackage{epstopdf}
\usepackage{amsfonts}
\usepackage{bm}
\begin{document}
Consider a convex region $\Omega$ in $\mathbb{R}^3$ which is symmetric about both the three axes. The origin is the center of the solid described by $\Omega$. Now consider a surface force exerted on it. The force $\bm{T}$(Neuton per surface unit)$=(T_1,T_2,T_3)$ has the following property:


\begin{tabular}{|c|c|c|}
\hline
$T_1(-x,y,z)=-T_1(x,y,z)$ & $T_1(x,-y,z)=T_1(x,y,z)$ & $T_1(x,y,-z)=T_1(x,y,z)$\\
\hline
$T_2(-x,y,z)=T_2(x,y,z)$ & $T_2(x,-y,z)=-T_2(x,y,z)$ & $T_2(x,y,-z)=T_2(x,y,z)$\\
\hline
$T_3(-x,y,z)=T_3(x,y,z)$ & $T_3(x,-y,z)=T_3(x,y,z)$ & $T_3(x,y,-z)=-T_3(x,y,z)$\\
\hline
\end{tabular}


To apply FEA(finite element analysis) on this solid, we can only consider $\frac{1}{8}$ of the region.
The fundumental equation of linear elasticity
\[
-(\lambda+G)\mathop{\nabla}(\mathop{\nabla}\cdot\textbf{u})-G\mathop{{}\bigtriangleup}\nolimits \textbf{u}=\textbf{F},
\]
contains only the second derivatives of coponents of $\bm{u}$. Therefore, if the equation holds for $(u_1(x,y,z),u_2(x,y,z),u_3(x,y,z))$ and it also holds for example $(-u_1(-x,y,z),u_2(-x,y,z),u_3(-x,y,z))$ (already verified). Because the solution is unique, the only possibility is
\begin{eqnarray*}
u_1(x,y,z)&=-u_1(-x,y,z)\\
u_2(x,y,z)&=u_2(-x,y,z)\\
u_3(x,y,z)&=u_3(-x,y,z)
\end{eqnarray*}


As a result, $u_1$ is odd function about $x$, $u_2,u_3$ is even function about $x$.
The other possibility is $(u_1(-x,y,z),-u_2(-x,y,z),-u_3(-x,y,z))$. The choice between the two dependes on the boundary condition symmetry about the x plane.

By Cauchy's stress formula, $\bm{\sigma}\cdot \bm{n}=\bm{T}$.
Since $n_1$ is odd function about x and the other two components are even function about x $\sigma_{11},\sigma_{22},\sigma_{33},\sigma_{23}$ is even function about x while $\sigma_{12},\sigma_{13}$ are odd function about x.
By some calculation, we can verify $(-u_1(-x,y,z),u_2(-x,y,z),u_3(-x,y,z))$ satisfies the boundary condition. and we can get:
\begin{eqnarray*}
u_1(0,y,z)=0\\
\frac{\partial u_2(x,y,z)}{\partial x}|_{x=0}=0\\
\frac{\partial u_3(x,y,z)}{\partial x}|_{x=0}=0
\end{eqnarray*}
$\tau_{xy}=(\frac{\partial u_1(x,y,z)}{\partial y}+\frac{\partial u_2(x,y,z)}{\partial x})|_{x=0}=0$ and similarly $\tau_{xz}=0$
In real calculation, natural boundary condition(Neumann,first order partial derivative) contributes nothing to integral of RHS of the weak form. And Dirichlet bounary condition adds enforced constraints to the assembled system matrix.
The rule for variable z is similar:$u_2(x,0,z)=0,u_3(x,y,0)=0$.
\end{document}
